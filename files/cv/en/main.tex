%%%%%%%%%%%%%%%%%%%%%%%%%%%%%%%%%%%%%%%%%
% Medium Length Professional CV
% LaTeX Template
% Version 2.0 (8/5/13)
%
% This template has been downloaded from:
% http://www.LaTeXTemplates.com
%
% Original author:
% Trey Hunner (http://www.treyhunner.com/)
%
% Important note:
% This template requires the resume.cls file to be in the same directory as the
% .tex file. The resume.cls file provides the resume style used for structuring the
% document.
%
%%%%%%%%%%%%%%%%%%%%%%%%%%%%%%%%%%%%%%%%%

%----------------------------------------------------------------------------------------
%	PACKAGES AND OTHER DOCUMENT CONFIGURATIONS 
%----------------------------------------------------------------------------------------
 
\documentclass{resume} % Use the custom resume.cls style 
\usepackage[dvipsnames]{xcolor}
\usepackage[colorlinks,linkcolor=red]{hyperref}
\usepackage{fontawesome}
\usepackage[left=0.4 in,top=0.3 in,right=0.4 in,bottom=0.3in]{geometry} % Document margins
\newcommand{\tab}[1]{\hspace{.2667\textwidth}\rlap{#1}}
\newcommand{\itab}[1]{\hspace{0em}\rlap{#1}}
\name{Huiqian Li} % Your name 
\address{\faMapMarker~Automotive Crash Laboratory, Tsinghua university, Haidian District, Beijing 100084, China} % Your address 
%\address{123 Pleasant Lane \\ City, State 12345} % Your secondary addess (optional) 
\address{{\faPhone}~ +86 199 3709 3760 ~~ {\faEnvelope}~\url{lihq20@mails.tsinghua.edu.cn}}
\address{{Github}:~\href{https://github.com/lhquixotic}{\faGithub}~~~Website:~\href{https://lhquixotic.github.io/}{\faHome}~ResearchGate:~~~\href{https://www.researchgate.net/profile/Li-Huiqian}{\faUser}} % Your phone number and email


\definecolor{TsinghuaPurple}{cmyk}{0.58,0.90,0,0}
\renewenvironment{rSection}[1]{
\sectionskip
\textcolor{TsinghuaPurple}{\MakeUppercase{#1}}
\sectionlineskip
\hrule
\begin{list}{}{
%\setlength{\leftmargin}{1.5em}
\setlength{\leftmargin}{0em}
}
\item[]
}{
\end{list}
}


\begin{document}  

%----------------------------------------------------------------------------------------
%	EDUCATION SECTION
%----------------------------------------------------------------------------------------

\begin{rSection}{Education}

{\bf \href{https://www.tsinghua.edu.cn/en/}{Tsinghua university}, Beijing, China} \hfill {August 2020 - Present}
\\ 
\textit{PhD candidate} in Mechanical Engineering, School of Vehicle and Mobility
\\
Supervised by \href{https://en.wikipedia.org/wiki/Zhong_Zhihua}{Zhihua Zhong} 

{\bf \href{https://english.bit.edu.cn/}{Beijing Institute of Technology}, Beijing, China} \hfill {August 2016 - July 2020}
\\ 
\textit{B.S.} in Automotive Engineering, School of Mechanical Engineering
\\
GPA: 92.4/100 Rank: 2/133

%Minor in Linguistics \smallskip \\
%Member of Eta Kappa Nu \\
%Member of Upsilon Pi Epsilon \\


\end{rSection} 

%----------------------------------------------------------------------------------------
%	Research Interests
%----------------------------------------------------------------------------------------

\begin{rSection}{Research INTERESTS}

\begin{rSubsection}{Trustworthy Artificial Intelligence in Autonomous Driving}{}{}

\item Safe and explainable AI decision-making algorithms
\item Evaluation of trustworthiness for AI decision-making
\item Model-based reinforcement learning 
\item Active inference theory and its application in autonomous driving
\end{rSubsection}  

\begin{rSubsection}{Decision-Making and Control in Autonomous Driving}{}{}

\item Pedestrian avoidance decision-making
\item Robust and adaptive path tracking control  
\item Autonomous driving platoon control
\end{rSubsection}  


\end{rSection}

%----------------------------------------------------------------------------------------
%	Technical Skills
%----------------------------------------------------------------------------------------

\begin{rSection}{skills}

\begin{tabular}{ @{} >{\bfseries}l @{\hspace{6ex}} l }  
Programming Languages & Python = Matlab/Simulink = C $>$ C++\\
Platform & Linux\\
Software & Pytorch, CARLA, CarSim/TruckSim, ROS, Cyber RT

\end{tabular}   

\end{rSection}

%--------------------------------------------------------------------------------------
%   Research Publications 
%--------------------------------------------------------------------------------------
\begin{rSection}{ Research Publication } \itemsep -3pt        
 {\textbf{Li H}, Huang J, \textit{et al.}. Adaptive robust path tracking control for autonomous vehicles with measurement noise. \textit{Int J Robust Nonlinear Control}. 2022; 32( 13): 7319– 7335. doi:10.1002/rnc.6218 \href{https://onlinelibrary.wiley.com/doi/10.1002/rnc.6218}{\faExternalLink}} \hfill May 2022 
 
{\textbf{Li H}, Huang J, \textit{et al.} Stochastic pedestrian avoidance for autonomous vehicles Using hybrid reinforcement learning. \textit{Front Inform Technol Electron Eng}. 2023; 24(1): 131-140. doi:10.1631/FITEE.2200128 \href{https://jzus.zju.edu.cn/iparticle.php?doi=10.1631/FITEE.2200128}{\faExternalLink} \hfill January 2023

{\textbf{Li H}, Tian J, \textit{et al.} Towards Trustworthy Decision-Making for Autonomous Vehicle: Survey and Challenges. (\textbf{\textit{Submitted}})}}
 
\end{rSection}

%-------------------------------------------------------------------------------
%	PROJECTS

\begin{rSection}{PROJECTS}

\begin{rSubsection}{Autonomous Driving Truck Platoon Control Algorithm Development} {June 2020 - August 2021}{}{}

\item Software platform building based on Baidu Apollo platform, decision-making and lateral and longitudinal control algorithms development.
\item Vehicle-to-vehicle (V2V) communication code development using Ultra Wide Band (UWB) device and V2X device.
\item Two-vehicle autonomous driving platoon test and the maximum speed up to 70km/h.
 
\end{rSubsection}  

%------------------------------------------------

\begin{rSubsection}{Autonomous Driving Express Vehicle Development}{October 2021 - January 2022}{}{} 
\item Software platform building based on ROS, path tracking control algorithm development.
\item Obstacle detection and avoidance based on LiDAR points.

\end{rSubsection}

\end{rSection} 


%	INTERNSHIP/TRAININGS 
%----------------------------------------------------------------------------------------

\begin{rSection}{INTERNSHIP/TRAININGS} \itemsep -3pt  

{\textbf{Autonomous Driving Platoon Development Internship}, \\ Tong-Tsing-Hu Collaborative Innovation Center, Qingdao International Academician Park,\\ Qingdao, Shandong, China}  \hfill  June 2020 - August 2021 \\
\textbf{\textit{Experience}}
Hardware platforms development and planning and control algorithms development based on Baidu Apollo platform using C++.
\\
{\textbf{Beijing Institute of Technology Driverless Racing Team}, 
\\Beijing Institute of Technology, Beijing, China} \hfill August 2018 - August 2020 \\     
\textbf{\textit{Experience}} Vehicle Control Unit (VCU) hardware development and control algorithm development through Matlab/Simulink code generation.

\end{rSection}  
 
 %-----------------------------------------------------------------------------
 % POSITION OF RESPONSIBILITY
 %-----------------------------------------------------------------------------
  
% \begin{rSection}{POSITION OF RESPONSIBILITY}

% \begin{rSubsection}{CAE and Powertrain Lead, Formula SAE}{August 2015 - Present}{GT Motorsports,a Formula Student Team of GTU}{}              
% \item Devised the design objectives and validation of designs through simulations and testings
% \item Concentrated on real time simulation of Exhaust System and the noise reduction of Exhaust system
% \item Part of core Design group in the team helping with various design decisions  
% \item Performed numerous simulations of various components of the car in the area of FEA and CFD segments with documentations   
% \end{rSubsection}  

%------------------------------------------------

% \begin{rSubsection}{Head coordinator of Mechanical section at Robotics club} {July 2015 - May 2016}{Sanjaybhai Rajguru College of Engineering}{} 
% \item A college level Robotics club established by students with the aim of learning and professional skill development \\among students and peers            
% \item Lead in Mechanical work of Robotics club, working mostly with CAD and Hardware systems
% \item Team leader and active member working to develop various robots of different concepts and configurations     
% \end{rSubsection}

% \end{rSection}
  

%--------------------------------------------------------------------------------------
% Extra-Cirrucular
%--------------------------------------------------------------------------------------

% \begin{rSection}{Extra-Cirrucular} \itemsep -2pt   

% \begin{itemize}
 
% \item STTP on \textbf{Life Long Research} under TEQIP-II, SVNIT, Surat \hfill February 2016 
% \item Participated in \textbf{Formula Student India}, An International FSAE competition,  
% \\Secured 9th rank overall \& 4th in Endurance \hfill January 2016  
% \item Seminar on \textbf{Introduction to Robotics and Arduino Programming}, SRCOE,Rajkot \hfill July 2015 
% \item \textbf{Junkyard}, BRIZINGER'15, a National Level Techfest,  GEC, Rajkot \hfill March-2015
% \item Seminar on \textbf{Rapid Prototyping}, COGNIZANCE 2K14, a National Level Technical Festival, \\CSPIT, Charotar  \hfill September-2014   
% \item \textbf{Rise of Machine}, PRAKARSH 9.0, a National Level Technical Symposium, SVIT, Vasad \hfill March-2014  
  
% \end{itemize}  


% \end{rSection} 


%---------------------------------------------------------------------------------
%  Honors/Awards
%--------------------------------------------------------------------------------


\begin{rSection}{Honors/Awards} \itemsep -2pt
{Excellent First-class Scholarship of Tsinghua University }\hfill {\em October 2022} \\
{Excellent Communist Youth League Members of Tsinghua University}\hfill {\em September 2021} \\
{Excellent Graduate in Beijing} \hfill {\em July 2020} \\
{Excellent Graduate of Beijing Institute of Technology}\hfill {\em July 2020}\\
{Champion of Formula Student Autonomous China (FSAC) }\hfill {\em October 2018}\\
{Second Prize in China Undergraduate Mathematical Contest in Modeling (\href{http://en.mcm.edu.cn/}{CUMCM})} \hfill {\em October 2018} \\
{Merit Student in Beijing} \hfill{\em May 2018}\\
{National Scholarship} \hfill{\em October 2017}
\end{rSection}


%---------------------------------------------------------------------------------
%  DECLARATION
%--------------------------------------------------------------------------------

\begin{rSection}{ Declaration  } \itemsep -3pt        

\item I hereby declare that all the details furnished above are true to the best of my knowledge and belief.   
  
\end{rSection}
\end{document}
